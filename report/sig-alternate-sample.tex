% This is "sig-alternate.tex" V2.1 April 2013
% This file should be compiled with V2.5 of "sig-alternate.cls" May 2012
%
% This example file demonstrates the use of the 'sig-alternate.cls'
% V2.5 LaTeX2e document class file. It is for those submitting
% articles to ACM Conference Proceedings WHO DO NOT WISH TO
% STRICTLY ADHERE TO THE SIGS (PUBS-BOARD-ENDORSED) STYLE.
% The 'sig-alternate.cls' file will produce a similar-looking,
% albeit, 'tighter' paper resulting in, invariably, fewer pages.
%
% ----------------------------------------------------------------------------------------------------------------
% This .tex file (and associated .cls V2.5) produces:
%       1) The Permission Statement
%       2) The Conference (location) Info information
%       3) The Copyright Line with ACM data
%       4) NO page numbers
%
% as against the acm_proc_article-sp.cls file which
% DOES NOT produce 1) thru' 3) above.
%
% Using 'sig-alternate.cls' you have control, however, from within
% the source .tex file, over both the CopyrightYear
% (defaulted to 200X) and the ACM Copyright Data
% (defaulted to X-XXXXX-XX-X/XX/XX).
% e.g.
% \CopyrightYear{2007} will cause 2007 to appear in the copyright line.
% \crdata{0-12345-67-8/90/12} will cause 0-12345-67-8/90/12 to appear in the copyright line.
%
% ---------------------------------------------------------------------------------------------------------------
% This .tex source is an example which *does* use
% the .bib file (from which the .bbl file % is produced).
% REMEMBER HOWEVER: After having produced the .bbl file,
% and prior to final submission, you *NEED* to 'insert'
% your .bbl file into your source .tex file so as to provide
% ONE 'self-contained' source file.
%
% ================= IF YOU HAVE QUESTIONS =======================
% Questions regarding the SIGS styles, SIGS policies and
% procedures, Conferences etc. should be sent to
% Adrienne Griscti (griscti@acm.org)
%
% Technical questions _only_ to
% Gerald Murray (murray@hq.acm.org)
% ===============================================================
%
% For tracking purposes - this is V2.0 - May 2012

\documentclass{sig-alternate-05-2015}
\setlength{\parindent}{2em} % indent size for chinese
\usepackage{indentfirst} % 中文第一段縮排套件
\usepackage[no-math]{fontspec}
\usepackage[SlantFont]{xeCJK}

\newcommand{\FontPath}{./font/}

\XeTeXinputencoding "utf8"
\XeTeXdefaultencoding "utf8"
\XeTeXlinebreaklocale "zh"
\XeTeXlinebreakskip = 0pt plus 1pt

\setmainfont[
    Mapping = tex-text,
    Path = \FontPath,
    UprightFont = *-Regular,
    BoldFont = *-Bold,
    BoldItalicFont = *-BoldItalic,
    ItalicFont = *-Italic,
]{NotoSerif}

\setsansfont[
    Mapping = tex-text,
    Path = \FontPath,
    UprightFont = *-Regular,
    BoldFont = *-Bold,
    BoldItalicFont = *-BoldItalic,
    ItalicFont = *-Italic,
]{NotoSans}

\setmonofont[
    Scale = 0.96,
    Path = \FontPath,
    UprightFont = *,
    BoldFont = *-Bold,
    BoldItalicFont = *-BoldItalic,
    ItalicFont = *-Italic,
]{Inconsolata-LGC}

\setCJKmainfont[
    Mapping = tex-text,
    Path = \FontPath,
    UprightFont = *-Regular,
    BoldFont = *-Bold,
]{NotoSansCJKtc}

\setCJKmonofont[
    Mapping = tex-text,
    Path = \FontPath,
    UprightFont = *-Regular,
    BoldFont = *-Bold,
]{NotoSansCJKtc}


\begin{document}

% Copyright
%\setcopyright{acmcopyright}
%\setcopyright{acmlicensed}
%\setcopyright{rightsretained}
%\setcopyright{usgov}
%\setcopyright{usgovmixed}
%\setcopyright{cagov}
%\setcopyright{cagovmixed}


% DOI
%\doi{10.475/123_4}

% ISBN
% \isbn{977-4567-24-567/08/06}

%Conference
%\conferenceinfo{PLDI '13}{June 16--19, 2013, Seattle, WA, USA}

%\acmPrice{\$15.00}

%
% --- Author Metadata here ---
%\conferenceinfo{WOODSTOCK}{'97 El Paso, Texas USA}
%\CopyrightYear{2007} % Allows default copyright year (20XX) to be over-ridden - IF NEED BE.
%\crdata{0-12345-67-8/90/01}  % Allows default copyright data (0-89791-88-6/97/05) to be over-ridden - IF NEED BE.
% --- End of Author Metadata ---

\title{Deep Political Leaning Analysis System (DPLAS)}
\subtitle{2016 Spring - Web Retrivel and Mining - Final Project}
%\titlenote{A full version of this paper is available as
%\textit{Author's Guide to Preparing ACM SIG Proceedings Using
%\LaTeX$2_\epsilon$\ and BibTeX} at
%\texttt{www.acm.org/eaddress.htm}}}
%
% You need the command \numberofauthors to handle the 'placement
% and alignment' of the authors beneath the title.
%
% For aesthetic reasons, we recommend 'three authors at a time'
% i.e. three 'name/affiliation blocks' be placed beneath the title.
%
% NOTE: You are NOT restricted in how many 'rows' of
% "name/affiliations" may appear. We just ask that you restrict
% the number of 'columns' to three.
%
% Because of the available 'opening page real-estate'
% we ask you to refrain from putting more than six authors
% (two rows with three columns) beneath the article title.
% More than six makes the first-page appear very cluttered indeed.
%
% Use the \alignauthor commands to handle the names
% and affiliations for an 'aesthetic maximum' of six authors.
% Add names, affiliations, addresses for
% the seventh etc. author(s) as the argument for the
% \additionalauthors command.
% These 'additional authors' will be output/set for you
% without further effort on your part as the last section in
% the body of your article BEFORE References or any Appendices.

\numberofauthors{4} %  in this sample file, there are a *total*
% of EIGHT authors. SIX appear on the 'first-page' (for formatting
% reasons) and the remaining two appear in the \additionalauthors section.
%
\author{
% You can go ahead and credit any number of authors here,
% e.g. one 'row of three' or two rows (consisting of one row of three
% and a second row of one, two or three).
%
% The command \alignauthor (no curly braces needed) should
% precede each author name, affiliation/snail-mail address and
% e-mail address. Additionally, tag each line of
% affiliation/address with \affaddr, and tag the
% e-mail address with \email.
%
% 1st. author
\alignauthor
林祐萱\\
\affaddr{B03902055}\\
       \affaddr{NTU CSIE}\\       
       \email{b03902055@ntu.edu.tw}
% 4th. author
\alignauthor 蔡尚佑\\
\affaddr{B03902068}\\
       \affaddr{NTU CSIE}\\       
       \email{b03902068@ntu.edu.tw}
% 2nd. author
\alignauthor
林書瑾\\
\affaddr{B03902078}\\
       \affaddr{NTU CSIE}\\       
       \email{b03902078@ntu.edu.tw}
% 3rd. author
\and
\alignauthor 陳力\\
\affaddr{B03902083}\\
       \affaddr{NTU CSIE}\\       
       \email{b03902083@ntu.edu.tw}
}
% There's nothing stopping you putting the seventh, eighth, etc.
% author on the opening page (as the 'third row') but we ask,
% for aesthetic reasons that you place these 'additional authors'
% in the \additional authors block, viz.
%\additionalauthors{Additional authors: John Smith (The Th{\o}rv{\"a}ld Group,
%email: {\texttt{jsmith@affiliation.org}}) and Julius P.~Kumquat
%(The Kumquat Consortium, email: {\texttt{jpkumquat@consortium.net}}).}
%\date{30 July 1999}
% Just remember to make sure that the TOTAL number of authors
% is the number that will appear on the first page PLUS the
% number that will appear in the \additionalauthors section.

\maketitle


%
% The code below should be generated by the tool at
% http://dl.acm.org/ccs.cfm
% Please copy and paste the code instead of the example below. 
%
%\begin{CCSXML}
%<ccs2012>
 %<concept>
  %<concept_id>10010520.10010553.10010562</concept_id>
  %<concept_desc>Computer systems organization~Embedded systems</concept_desc>
  %<concept_significance>500</concept_significance>
 %</concept>
 %<concept>
  %<concept_id>10010520.10010575.10010755</concept_id>
  %<concept_desc>Computer systems organization~Redundancy</concept_desc>
  %<concept_significance>300</concept_significance>
 %</concept>
 %<concept>
  %<concept_id>10010520.10010553.10010554</concept_id>
  %<concept_desc>Computer systems organization~Robotics</concept_desc>
  %<concept_significance>100</concept_significance>
 %</concept>
 %<concept>
  %<concept_id>10003033.10003083.10003095</concept_id>
  %<concept_desc>Networks~Network reliability</concept_desc>
  %<concept_significance>100</concept_significance>
 %</concept>
%</ccs2012>  
%\end{CCSXML}

%\ccsdesc[500]{Computer systems organization~Embedded systems}
%\ccsdesc[300]{Computer systems organization~Redundancy}
%\ccsdesc{Computer systems organization~Robotics}
%\ccsdesc[100]{Networks~Network reliability}


%
% End generated code
%

%
%  Use this command to print the description
%
%\printccsdesc

% We no longer use \terms command
%\terms{Theory}

\section{Introduction}
The main objective is to analyze a user’s political leaning based on his posts. With vector space model, we treat query (user’s posts) and documents (political parties’ posts) as vectors. Then we can analyze user’s political leaning with distance or similarity between their vectors.

Another important assumption is that posts about different topic may come from different language model. If we treat them as a large vector, it’s unreasonable that a post can contain various topics. As a result, we will classify documents and user’s query into different topics, creating a vector space model in each topic.

\section{Implementation}
The whole algorithm is basically a multi-class classification algorithm, with the political parties as classes. Most of time, text classification problem is simply applied with some linear classification techniques e.g. linear SVM. But here we think that since each document has its own topic like talking about gay-marriage.

The assumption we make here is that documents about different topic may come from different language model. So what we do first is to partition our documents into different topics by clustering algorithm. Then we train classifiers for each different topic base on the documents of the topic.

For a query (a user) , we treat it as a set of documents. Then we classify each document of that set (user) , aggregate the result and turns it into the result for that user.

\subsection{Cluster Data Into Different Topics}

The first job we have to do is to cluster data into different topics. We use K-Means Clustering to separate them into 12 different topics. The motivation of clustering data is: for documents with various topics, we assume they are generated from different language models.

For instance, a document about homosexuality is irrelative to capital punishment, and we’ll assume it is generated from homosexuality model.

\subsection{Build Bigram Vector Space Model}

For each topic, we build a bigram vector space model based on data in the cluster. Thus, for a new document in this topic, we can measure it as a vector.

This model is based on \textit{TF-IDF} and \textit{Okapi BM25} smoothing method.

\large
\begin{itemize}
   \item  $TF_{doc}(term) = \log(\frac{total\ docs-docfreq(term) + 0.5}{docfreq(term) + 0.5})$


\item $IDF_{doc}(term) = \frac{c(term,doc)(k+1)}{c(term,doc)+k(1-b+b \times \frac{doc}{avgdoclen})}$

\item $TFIDF_{doc}(term) = TF_{doc}(term) \times IDF_{doc}(term)$
\end{itemize}
\normalsize
where $c(term, doc)$ is the term frequency in document $doc$.

In this project, the parameters $(k, b)$ are $(1.6, 0.75)$.


\subsection{Measure Posts From Known Parties}

For a post from a known party, we first determine which topics it discusses about, and measure it in the corresponding vector space model. This vector will be labeled to the specific party.

As a result, there are lots of labeled vectors in each topic model. They will be used to analyze user's political leanings.



\begin{table*}
\centering
\caption{Result}

\begin{tabular}{|l|l|l|}
\hline
Name        & Party                  & Result(國民黨,民進黨,時代力量,親民黨,綠黨,社民黨)                  \\ \hline
國民黨李大砲 李新   & 國民黨                    & \textbf{0.357},0.053,0.185,0.125,0.126,0.155 \\ \hline
民主進步黨       & 民進黨                    & \textbf{0.298},0.065,0.218,0.131,0.131,0.156 \\ \hline
柯文哲         & 無黨籍                    & \textbf{0.260},0.075,0.244,0.129,0.129,0.162 \\ \hline
蕭美琴         & 民進黨                    & 0.236,0.118,\textbf{0.238},0.126,0.127,0.155 \\ \hline
連勝文         & 國民黨                    & \textbf{0.366},0.035,0.234,0.116,0.116,0.133 \\ \hline
邱毅『談天論地話縱橫』 & 新黨(親民-國民黨) & 0.174,0.188,\textbf{0.218},0.130,0.131,0.159 \\ \hline
\end{tabular}
\end{table*}

\subsection{Analyze User Political Leanings}

To analyze a user's political leaning, we first have to extract the user's information. Here we treat each user as a set of vectors $\{d\}$ , where each vector representing a post/article/document he wrote.

Then we use our algorithm above to estimate the probability $Pr( x \text{ is similar to posts of political party } y \| x , \theta )$ of each post it wrote. But it remains an issue \: what is the political leaning of that user given his post's political bias ?

Here we just average all political bias of its posts and treat it as the political leaning of that user. Next, we apply \textit{K-Nearest Neighbors} algorithm to find k-nearest vectors (\textit{logistic regression} is also used) with \textit{cosine similarity}. Finally, we make k-nearest vectors to "vote" political leanings of user's posts.

But one more issue remains : what if that user don't post a lot?

Here we use an active-learning-flavored idea : provide several posts from out data set and make the user to choose which it identify with. The post the user choose can be seen as the post it wrote. And the extended user posts set is then applied with our algorithm to figure out the political leaning.

\section{Result}

The ground truth of a person's political leanings is hard to define. Hence, we estimate legislators' political leanings of different known parties. The results shown in \textit{table 1} are estimated probability of having political leaning similar to some party. The order is 國民黨, 民進黨, 時代力量, 親民黨, 綠黨, 社會民主黨 respectively.

We found in most case, if the user input is sufficient and highly related to politics, the political leanings we predict is very precise. However, if user tends to post neutral and ambiguous remarks, the precision will be limited. When user's posts are unbiased, there are too many similar documents.

\section{Work Distribution}

\begin{itemize}
  \item 陳力  : model design , data clustering , website data crawling
  \item 林祐萱: demo website , FB posts data crawling , inverted file , model design 
  \item 林書瑾: inverted file, topic VSM, \textit{Liberty Times} news crawling
  \item 蔡尚佑: PTT Data crawling
\end{itemize}

\section{Document Set}

We crawl data from:

\begin{itemize}
 \item  Liberty Times News, UDN news 
 \item  Facebook Pages
 \item  PTT Gossiping
 \item  Political Parties and NGOs
\end{itemize}

Download Link:

https://goo.gl/69bUCH

\end{document}
